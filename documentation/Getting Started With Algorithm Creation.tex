\documentclass[11pt]{article}
%Gummi|065|=)
\usepackage{listings}
\usepackage{color}

\definecolor{dkgreen}{rgb}{0,0.6,0}
\definecolor{gray}{rgb}{0.5,0.5,0.5}
\definecolor{mauve}{rgb}{0.58,0,0.82}

\lstset{frame=tb,
  language=Java,
  aboveskip=3mm,
  belowskip=3mm,
  showstringspaces=false,
  columns=flexible,
  basicstyle={\small\ttfamily},
  numbers=none,
  numberstyle=\tiny\color{gray},
  keywordstyle=\color{blue},
  commentstyle=\color{dkgreen},
  stringstyle=\color{mauve},
  breaklines=true,
  breakatwhitespace=true
  tabsize=3
}
\title{\textbf{An Introduction To Writing Algorithms \emph{for}\\Group Alpha Testing Suite}}
\author{Adam Gleave\\
		Nick Burscough\\
		Artjoms Iskovs\\
		Lawrence Esswood\\
		Christopher Little\\
		Alan Rusnak}
\date{}
\begin{document}

\maketitle

\section{Getting Started}

Version 1.0 of the Testing Suite is almost entirely written in Java. As such, all algorithms that are to be submitted to the system must currently also be written in Java. This example documentation will remain largely independent of IDE, though Eclipse is a good choice due to its ability to streamline JAR creation (as algorithms are submitted to the system as JAR files).\\
To dive right into algorithm creation, we will now look through the workings of a hello-world-esque algorithm. The algorithm works in a completely random fashion, buying and selling according to the whims of its lord and master, \texttt{math.random()}. Following is the algorithm in its entirety, after which we will take a look at a breakdown of how it interfaces with the simulated market.\newpage

\begin{lstlisting}
// HelloRiches.java

import java.util.Iterator;
import orderBooks.Order;
import orderBooks.OrderBook;
import database.StockHandle;
import testHarness.ITradingAlgorithm;
import testHarness.MarketView;
import testHarness.clientConnection.Options;


public class HelloRiches implements ITradingAlgorithm {

	@Override
	public void run(MarketView marketView, Options options) {
		Iterator<StockHandle> stocks = marketView.getAllStocks().iterator();
		StockHandle bestStockEver = stocks.next();
		OrderBook book = marketView.getOrderBook(bestStockEver);
		double woot = 4;
		while(!marketView.isFinished()) {
			marketView.tick();
			woot += Math.random()-0.5;
			if(woot < 5) {
				//BUY BUY BUY
				Iterator<? extends Order> iter = book.getAllOffers();
				if(!iter.hasNext()) continue;
				int bd = iter.next().getPrice();
				marketView.buy(bestStockEver, bd, 1);
			} else {
				//SELL SELL SELL
				Iterator<? extends Order> iter = book.getAllBids();
				if(!iter.hasNext()) continue;
				int bd = iter.next().getPrice();
				marketView.sell(bestStockEver, bd, 1);
			}
			
		}
	}
}

\end{lstlisting}

\section{Breakdown of HelloRiches.java}
\subsection{Imports}
\begin{lstlisting}
import orderBooks.Order;
import orderBooks.OrderBook;
\end{lstlisting}
The orderbook construct holds all bids and offers for a stock. Each stock has an associated order book. An order is simply one of these bids or offers. We will see later how we can use these classes to place our algorithm's offers onto the market and also bid for existing orders.\\
\begin{lstlisting}
import database.StockHandle;
import testHarness.ITradingAlgorithm;
import testHarness.MarketView;
import testHarness.clientConnection.Options;
\end{lstlisting}
A stock handle essentially does what it says on the tin and allows us to hold a reference to a particular stock so that we interact with it (buying or selling etc). Market View and Options exist to allow the testing framework to test market situations and vary simulation options such as tick size. Finally, the ITrandingAlgorithm class is implemented by the algorithm to ensure correct interfacing with the framework (i.e. the presence of a \texttt{run()} method.\\

\begin{lstlisting}
@Override
public void run(MarketView marketView, Options options) {
	Iterator<StockHandle> stocks = marketView.getAllStocks().iterator();
	StockHandle bestStockEver = stocks.next();
	OrderBook book = marketView.getOrderBook(bestStockEver);
\end{lstlisting}

\section{Key Commands/Concepts}
We hope you will enjoy using this release as much as we enjoyed creating it. If you have comments, suggestions or wish to report an issue you are experiencing - contact us at: \emph{http://gummi.midnightcoding.org}.

\section{A (more in depth) example}
If you are wondering where your old default text is; it has been stored as a template. The template menu can be used to access and restore it. 

\end{document}
